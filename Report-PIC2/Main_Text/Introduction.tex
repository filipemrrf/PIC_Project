When studying gravitational waves, one has to keep in mind that gravitational wave sources (like black hole mergers, neutron stars, etc.) are usually located billions of light years away from us. From the point of view of an observer on Earth, this distance is so vast that it can be approximated as being at infinity. This distance, however, is not sufficiently far away to require accounting for the cosmological constant, allowing us to model spacetime as asymptotically flat. Thus, we are interested in studying the behavior of the gravitational fields at null infinity, $\mathscr{I}$.

To do so, we must perform a conformal compactification of the spacetime, which brings $\mathscr{I}$ to a finite distance on our computational grid. This is done by working in the hyperboloidal coordinate system.

\subsection{Hyperboloidal Coordinates}

The hyperboloidal coordinate system, as previously mentioned, maps our previously unbonded domain to a finite one. This is done by introducing new time and radial coordinates $(t,r)$, which are related to the spherical coordinates of Minkowski spacetime $(T,R)$ by the following transformations:

\begin{equation}
    \begin{array}{c c} 
        t = T - H(R) & R = \frac{r}{\Omega(r)} \; ,
    \end{array} \; 
\end{equation}
%
where $H(R)$ is called the height function and $\Omega(r)$ is called the compress function. Throughout this work, we will use the following choices for these functions:

\begin{equation}
    \begin{array}{c c}
        H(R) = \sqrt{S^2+R^2} & \Omega(r) = \frac{1}{2} \left(1 - \frac{r^2}{S^2}\right)
    \end{array} \; ,
\end{equation}
%
where $S$ is a constant that determines the size of the compactified domain. The choice of $S$ is arbitrary as it simply defines what point $\mathscr{I}$ will be maped to. In this work, we will use $S = 1$.

This coordinate transformation gives rize to the following Jacobian matrix:

\begin{equation}
    \left(J^{Hyp}\right)_{\alpha'}^{\ \ \beta} = 
    \begin{pmatrix}
        1 & -H'(r) & 0 & 0 \\
        0 & \frac{L(r)}{\Omega^2(r)} & 0 & 0 \\
        0 & 0 & 1 & 0 \\
        0 & 0 & 0 & 1
    \end{pmatrix} \; ,
\end{equation}
%
where $H'(r)$ denotes the derivative of the height function with respect to $R$ written as a function of $r$, and $L(r)$ is defined as
%
\begin{equation}
    L(r) \equiv \Omega(r) - r \, \partial_r \Omega(r) \; .
\end{equation}

For the previously chosen height and compress functions, we have
%
\begin{equation}
    \begin{array}{c c}
        H'(r) = \frac{2 \, r \, S}{S^2 + r^2} & L(r) = \frac{1}{2} \left(1 + \frac{r^2}{S^2}\right) \; .
    \end{array}
\end{equation}


\subsection{Computational Setup}

This work is a continuation of my previous work on numerical relativity. As such, the framework used here is the same as the one used in that work, with the adition of truncation error matching for the derivatives, interpolation at the boundaries and the Evans Method for regularization at the origin. The description of the base code can be found in \cite{}, and this section will elaborate on the upgrades made.

\subsubsection{Truncation Error Matching}

Truncation error matching is a technique used to improve the accuracy of the numerical solution by matching the truncation error of the finite difference scheme used on the boundaries to the truncation error of the one used in the interior of our computational domain. This is done by using a one sided finite difference scheme on the boundaries such that the leading order error term is the same as the one used in the interior.

In our framework, we use the following second order finite difference scheme for the first derivative of a field $\psi$ at an interior point $i$ (where the leading order error term was written explicitly):

\begin{equation}
    \psi'_i = \frac{\psi_{i+1} - \psi_{i-1}}{2h} - \frac{h^2}{6} \psi'''_i + ...\; ,
\end{equation}
%
where $\psi_{i+1}$ and $\psi_{i-1}$ are the values of the field $\psi$ at the points $i+1$ and $i-1$ respectively, and $h$ is the grid spacing.

To match this leading order term of the error, we use the following one sided finite difference scheme for the derivative of $f$ at the left and right boundary points respectively \cite{}:

\begin{equation}
    \psi'_i = \frac{\psi_{i+3} - 4 \psi_{i+2} + 7 \psi_{i+1} - 4 \psi_{i}}{2h} - \frac{h^2}{6} \psi'''_i + ...\;
\end{equation}

\begin{equation}
    \psi'_i = \frac{4 \psi_{i} - 7 \psi_{i-1} + 4 \psi_{i-2} - \psi_{i-3}}{2h} - \frac{h^2}{6} \psi'''_i + ...\;
\end{equation}

\subsubsection{Interpolation at the Boundaries}

Since we are interested in evolving the fields at null infinity, we must choose a boundary condition that allows the fields to freely propagate outwards. To do so, we use interpolation at the outer boundaries of our computational domain to fill the ghost points in those regions. This is done by using the following interpolation scheme:

\begin{equation}
    \psi_i = 4 \psi_{i-1} - 6 \psi_{i-2} + 4 \psi_{i-3} - \psi_{i-4} \; ,
\end{equation}
%
where $\psi_i$ is the value of the field at the ghost point $i$, and $\psi_{i-1}$, $\psi_{i-2}$, $\psi_{i-3}$ and $\psi_{i-4}$ are the values of the field at the points $i-1$, $i-2$, $i-3$ and $i-4$ respectively.

\subsubsection{Evans Method}
When dealing with operators like the Laplacian in spherical coordinates, we find some formal singularities which need to be removed in order for our code to work. To remove those singularities, we can apply the Evans Method. This method consists of rewriting the singular terms as a different differential operator, called the Evans operator, which can be evaluated at the grid points. The Evans operator is defined as

\begin{equation}
    \partial_r \psi + \frac{p}{r}\psi = (p+1) \frac{d(r^p \psi)}{dr^{p+1}}\;,
\end{equation}
%
where $p$ is a constant. This operator can be expressed in terms of the grid points as 

\begin{equation}
    (p+1) \frac{d(r^p \psi)}{dr^{p+1}}=(\tilde{D}\psi)_i = (p+1)\frac{r^p_{i+1}\psi_{i+1}-r^p_{i-1}\psi_{i-1}}{r^{p+1}_{i+1}-r^{p+1}_{i-1}}\;,
\end{equation}
%
where the subscripts $i+1$ and $i-1$ denote the grid points $i+1$ and $i-1$ respectively.