As seen previously, when using hyperboloidal coordinates we expect incoming waves to be hard to resolve, as the error assiciated with them is expected to grow very rapidly as the wave prepagates. We will try to quantify that error by using incoming waves as initial conditions and comparing the numerical solution with the exact solution. It is then possible to study how much the error grows with the distance traveled towards the origin by doing several runs with the starting pulse further and further away.

First, we must build initial conditions that are purely incoming. We do that by doing a linear combination of an incoming and an outgoing wave, while making sure that the solution decays with $1/R$, that is

\begin{equation}
    \psi(T,R) = \frac{f(T+R) - f(T-R)}{R},
\end{equation}
%
where we can freely choose the function $f$. By choosing $f$ in such a way that in our domain, the outgoing wave vanishes. We will therefore choose $f$ to be a gaussian pulse given by

\begin{equation}
    f(x) = e^{-C (x - x_0)^2/2},
\end{equation}
%
where $x_0$ shiftes the center of the pulse and $C$ determines the width of the pulse and we will consider $C = 100$. Thus, the full set of initial conditions is 

\begin{equation}
    \begin{array}{c c c}
        \psi(T,R) = \frac{f(T+R) - f(T-R)}{R} & \Phi(T,R) = -\frac{f'(T+R) - f'(T-R)}{R} & \Pi(T,R) = \frac{\left(f'(T+R) - f'(T-R)\right)R - \left(f(T+R) - f(T-R)\right)}{R^2}
    \end{array} \; ,
    \label{eq:Incoming_initial_conditions}
\end{equation}
%
where $f'$ is the derivative of $f$. We now apply the change to hyperboloidal coordinates (where we defined a slightly modified height function $H(R) = \sqrt{S^2 + R^2} - S$ in order to make $t$ and $T$ coincide at $R=0$) and run several simulations for the wave equation in 3+1 dimensions with spherical symmetry (without applying the rescaling of the fields) up until the center of the pulse reaches the origin, while ranging $x_0$ from 1 to 10. The initial conditions for each pulse and the variation in time of the norm of the exact error normalized to the norm of the solution can be found in figure \ref{fig:Incoming_Waves}. 

\begin{figure}[h]
    \centering
    \begin{subfigure}[b]{0.45\textwidth}
        \centering
        \includegraphics[width=\textwidth]{Images/Incoming_IC.png}
    \end{subfigure}
    \hfill
    \begin{subfigure}[b]{0.45\textwidth}
        \centering
        \includegraphics[width=\textwidth]{Images/Incoming_Error.png}
    \end{subfigure}
    \caption{Evolution of the cubic wave equation in 3+1 dimensions with spherical symmetry using hyperboloidal coordinates with the initial conditions given in equation \eqref{eq:cubic_wave_equation-2nd_order_initial_conditions}. On the left, we have $A=1$ and $C=10$. On the right, we have  $A=10$ and $C=10$.}
    \label{fig:Incoming_Waves}
\end{figure}

We can see just by the initial conditions that, in fact incoming waves become harder to resolve the further away they are, as in our initial conditions the gaussian gets thinner as it approaches $\mathscr{I}$. Addicionally, we can see that the error increases very rapidly as the waves propagate. Both these results were what we would expect to happen for this coordinate system. Therefore, if on our system we expect to have incoming waves, we should modify the coordinates to have a traditional Cauchy slice for part of the domain and then join that slice to a hyperboloidal slice when we don't expect to have incoming waves anymore.