Throughout this work, hyperboloidal coordinate system proved to be a very useful tool when studying the behaviour of outgoing waves, compactifying our domain while keeping the characteristic speeds of the waves finite. Our numerical experiments demonstrated the reliability of our code framework when dealing with different variations of the wave equation in this coordinate system, by showing clean second order convergence in all the performed simmulations.

However, this choice of coordinates is not adequate when there are incoming waves in our problem, since they are very hard to resolve and are great sources of error in the simulations. This suggests that a hybrid approach may be ideal for the cases where we expect to have incoming waves in a small portion of our domain, incorporating traditional Cauchy slices alongside hyperboloidal ones.

The master thesis that follows this work, will be focused on expanding the well estalished numerical relativity code \textit{BAMPS}, making it able to perform simmulations in this coordinate system. In addition, more intricate non linearities will be studied, as we try to get closer to being able to solve the Einstein equations numerically.