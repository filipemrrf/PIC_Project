Throughout this work, the hyperboloidal coordinate system proved to be a very advantageous and convenient tool when studying the behavior of outgoing waves. Our numerical experiments demonstrated the reliability of our code framework when dealing with different variations of the wave equation in these coordinates by showing clean second-order convergence in all the performed simulations.

However, this choice of coordinates is not adequate when there are incoming waves in our problem since they are very hard to resolve and are great sources of error in the simulations. This suggests that a hybrid approach may be ideal for the cases where we expect to have incoming waves in a small portion of our domain, incorporating traditional Cauchy slices alongside hyperboloidal ones.

The master thesis that follows this project will focus on expanding the well-established numerical relativity code \texttt{BAMPS}, making it able to perform simulations in this coordinate system. In addition, we will look into more intricate nonlinearities with both mathematical and physical significance such as the wave equation with quadratic nonlinearity, as we try to get closer to solving the Einstein equations numerically.