It is finally time to tackle non-linear variations of the wave equation in 3+1 dimensions with spherical symmetry. We start by writing the wave equation as
%
\begin{equation}
    \Box \psi \equiv - \partial_T^2 \psi + \frac{1}{R^2} \partial_R\left( R^2 \partial_R \psi\right) = \psi^3 \;.
\end{equation}
%
By applying a first-order reduction followed by the coordinate change to hyperboloidal coordinates and setting $\gamma_2=0$, we get
%
\begin{equation}
    \left\{ \begin{array}{l} 
        \partial_T \Psi = - \Pi \\ 
        \partial_T \Phi = \mathcal{B}\left((r^2 + \Omega^2)^2 \left(H' \partial_r \Phi + \partial_r\Pi\right) + H' L \Omega \left( 2r\Phi - 3 \Omega \Psi + 2 r^{-1} \Omega^2 \Phi\right)\right) + \frac{\mathcal{A} L H'}{r^2 + \Omega^2} \Psi^3\\
        \partial_T \Pi = \mathcal{B}\left((r^2 + \Omega^2)^2 \left(\partial_r \Phi + H' \partial_r\Pi\right) + L \Omega \left( 2r\Phi - 3 \Omega \Psi + 2 r^{-1} \Omega^2 \Phi\right)\right) + \frac{\mathcal{A} L}{r^2 + \Omega^2} \Psi^3
        \end{array} \right. \; ,
\end{equation}
%
where we used the previous definitions for $\mathcal{A}$ and $\mathcal{B}$. 

Differently from the linear case, where we expect Gaussian initial conditions that only differ by amplitude to behave similarly, we reckon that parameter highly influences the solution. We will investigate this by solving the cubic wave equation with different initial conditions and comparing the results. Using the same boundary conditions as we did previously and using the initial conditions
%
\begin{equation}
    \begin{array}{c c c c}
        \psi(0,r) = A \, e^{-C \, r^2/2} \; , & \Phi(0,r) = - A \,C \, r \, \frac{\Omega^2(r)}{L(r)} \, e^{-C \, r^2/2} & \text{and} & \Pi(0,r) = 0
    \end{array} \; ,
    \label{eq:cubic_wave_equation-2nd_order_initial_conditions}
\end{equation}
%
with $A = 1.0$ for one of the runs and $A = 10.0$ for the other, while keeping $C = 10$ we get the results shown in figure \ref{fig:cubic_wave_eq}. We can see that, for $A = 1$, we get a solution that behaves very similarly to the linear one, as the wave starts to disperse. However, compared to the linear solution, it takes longer to disperse as there is a source term. For $A = 10$, we get a solution that grows rapidly until it explodes.

\begin{figure}[h]
    \centering
    \begin{subfigure}[b]{0.45\textwidth}
        \centering
        \includegraphics[width=\textwidth]{Images/Cubic_Wave_Equation_3+1_Spherical-A=1-Solution.png}
    \end{subfigure}
    \hfill
    \begin{subfigure}[b]{0.45\textwidth}
        \centering
        \includegraphics[width=\textwidth]{Images/Cubic_Wave_Equation_3+1_Spherical-A=10-Solution.png}
    \end{subfigure}
    \caption{Evolution of the cubic wave equation in 3+1 dimensions with spherical symmetry using hyperboloidal coordinates with the initial conditions given in equation \eqref{eq:cubic_wave_equation-2nd_order_initial_conditions}. On the left, we have $A=1$ and $C=10$. On the right, we have  $A=10$ and $C=10$.}
    \label{fig:cubic_wave_eq}
\end{figure}

Despite behaving differently in both cases, our solutions show similar convergence. For $A=1$, our solution converges cleanly in the norm and pointwise convergence tests, as shown in figure \ref{fig:cubic_wave_eq_convergence-1}, which is to be expected due to the similarity to the linear solution and our previous results. For $A = 10$, we obtain the same second-order convergence until the analytical blowup, as shown in figure \ref{fig:cubic_wave_eq_convergence-10}. 

\begin{figure}[h]
    \centering
    \begin{subfigure}[b]{0.45\textwidth}
        \centering
        \includegraphics[width=\textwidth]{Images/Cubic_Wave_Equation_3+1_Spherical-A=1-Norm.png}
    \end{subfigure}
    \hfill
    \begin{subfigure}[b]{0.45\textwidth}
        \centering
        \includegraphics[width=\textwidth]{Images/Cubic_Wave_Equation_3+1_Spherical-A=1-Pointwise.png}
    \end{subfigure}
    \caption{Convergence tests of the evolution of the cubic wave equation in 3+1 dimensions with spherical symmetry using hyperboloidal coordinates, provided the initial conditions given in equation \eqref{eq:cubic_wave_equation-2nd_order_initial_conditions}, with $A=1$ and $C=10$. On the left, we have the $L^2$ norm convergence, and on the right, we have the pointwise convergence at $\mathscr{I}$.}
    \label{fig:cubic_wave_eq_convergence-1}
\end{figure}

\begin{figure}[h]
    \centering
    \begin{subfigure}[b]{0.45\textwidth}
        \centering
        \includegraphics[width=\textwidth]{Images/Cubic_Wave_Equation_3+1_Spherical-A=10-Norm.png}
    \end{subfigure}
    \hfill
    \begin{subfigure}[b]{0.45\textwidth}
        \centering
        \includegraphics[width=\textwidth]{Images/Cubic_Wave_Equation_3+1_Spherical-A=10-Pointwise.png}
    \end{subfigure}
    \caption{Convergence tests of the evolution of the cubic wave equation in 3+1 dimensions with spherical symmetry using hyperboloidal coordinates, provided the initial conditions given in equation \eqref{eq:cubic_wave_equation-2nd_order_initial_conditions}, with $A=10$ and $C=10$. On the left, we have the $L^2$ norm convergence, and on the right, we have the pointwise convergence at $\mathscr{I}$.}
    \label{fig:cubic_wave_eq_convergence-10}
\end{figure}